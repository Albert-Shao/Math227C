\documentclass[12pt,letterpaper]{article}
\usepackage{amsmath,amsthm,amsfonts,amssymb,amscd}
\usepackage{fullpage}
\usepackage{lastpage}
\usepackage{enumerate}
\usepackage{fancyhdr}
\usepackage{mathrsfs}
\usepackage[margin=3cm,bottom=6cm]{geometry}
\usepackage{wrapfig}
\usepackage{graphicx}

\setlength{\parindent}{0.0in}
\setlength{\parskip}{0.05in}

\renewcommand{\theenumi}{\bf\Alph{enumi}}

% Edit these as appropriate
\newcommand\course{Math 227C}
\newcommand\semester{Spring 2019}     % <-- current semester
\newcommand\hwnum{3}                  % <-- homework number
\newcommand\yourname{Jun Allard} % <-- your name
%\newcommand\login{jcarberr}           % <-- your CS login

\newenvironment{answer}[1]{
  \subsubsection*{Problem \hwnum.#1}
}{\newpage}

\pagestyle{fancyplain}
\headheight 35pt
\lhead{ Math 227C}
\chead{\textbf{ Problem Set 6}}
%\rhead{Due {\bf Friday, May 11th}}
\headsep 20pt

\begin{document}
 

\begin{enumerate}





%%%%%%%%%%%%%% PROBLEM %%%%%%%%%%%%%%%%%%


%%%%%%%%%%%%%% PROBLEM %%%%%%%%%%%%%%%%%%
\item Many processes, including the spread of an infectious disease through a small community, can be modeled as first-order exponential processes like 
\begin{equation*}
\frac{dY}{dt} = \left(R-1\right) Y \quad Y(0)=1
\end{equation*}
where $R$ is a constant. This will either lead to exponential growth or exponential decay.

Assume instead that $R$ is a random variable that is different for each community. Assume it has Gaussian distribution with mean 1 and standard deviation $\sigma$,
\begin{equation*}
p_R(r) = \frac{1}{\sqrt{2\pi \sigma^2}}\, e^{-\left(r-1\right)^2/2\sigma^2}.
\end{equation*}

Find the probability density function $p_Y(y,t)$ of $Y(t)$.

Intuitively, we expect half of the trajectories to grow exponentially, and half of the trajectories to decay exponentially. 

Sketch the probability density you found for $p_Y(y,t)$. Does it agree with your intuition?


%%%%%%%%%%%%%%%%%%%%%%%%%%%%%%%%%%%%%% 
\end{enumerate}
\end{document}

%%%%%%%%%%%%%%%%%%%%%%%%%%%%%%%%%%%%%% 

