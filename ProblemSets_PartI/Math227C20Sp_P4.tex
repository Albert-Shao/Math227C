\documentclass[12pt,letterpaper]{article}
\usepackage{amsmath,amsthm,amsfonts,amssymb,amscd}
\usepackage{fullpage}
\usepackage{lastpage}
\usepackage{enumerate}
\usepackage{fancyhdr}
\usepackage{mathrsfs}
\usepackage[margin=3cm,bottom=6cm]{geometry}
\usepackage{wrapfig}
\usepackage{graphicx}

\setlength{\parindent}{0.0in}
\setlength{\parskip}{0.05in}

\renewcommand{\theenumi}{\bf\Alph{enumi}}


% Edit these as appropriate
\newcommand\course{Math 227C}
\newcommand\semester{Spring 2019}     % <-- current semester
\newcommand\hwnum{2}                  % <-- homework number
\newcommand\yourname{Jun Allard} % <-- your name
%\newcommand\login{jcarberr}           % <-- your CS login

\newenvironment{answer}[1]{
  \subsubsection*{Problem \hwnum.#1}
}{\newpage}

\pagestyle{fancyplain}
\headheight 35pt
\lhead{ \course\  }
\chead{\textbf{ Problem Set 4}}
%\rhead{Due {\bf Friday, April 27th}}
\headsep 20pt

\begin{document}


\begin{enumerate}


%%%%%%%%%%%%%% PROBLEM %%%%%%%%%%%%%%%%%%
\item Your advisor has scheduled two appointments with two graduate students, one at 1pm and the other at 1:30pm. The amounts of time that appointments last are independent exponential random variables with mean of 30 minutes. Assuming both graduate students arrive on time, find the expected amount of time that the 2nd grad student spends waiting outside the advisor's office (and/or Zoom meeting waiting room).


%%%%%%%%%%%%%% PROBLEM %%%%%%%%%%%%%%%%%%
\item A viral linear DNA molecule of length $l$ is known to contain a certain marked position, with the exact location of this mark being unknown. One approach to locating the marked position is to cut the molecule by agents that break it at points chosen according to a Poisson process with rate $\lambda$. (Note this is a spatial rate, with units of 1/length.) It is then possible to determine the fragment that contains the marked position. For instance, letting $m$ denote the location on the line of the marked position, then if $L_1$ denotes the last Poisson event to the left of $m$ (or $0$ if there are no Poisson events in $[0,m]$), and $R_1$ denotes the first Poisson event after $m$ (or $l$ if there are no Poisson events in $[m,l]$), then it would be learned that the marked position lies between $L_1$ and $R_1$.

Set $l=1$. Find
\begin{enumerate}[i.]
\item $P(L_1=0)$
\item $P(L_1<x)$, $0<x<m$
\item $P(R_1=1)$
\item $P(R_1> x)$, $m<x<1$
\end{enumerate}
By repeating the process with many identical copies of the DNA molecule, we can locate the marked position. If the process is repeated $n$ times on $n$ identical copies of DNA, yielding $L_i$, $R_i$, $i=1,...n$, then the marked position lies between $L$ and $R$ where
\begin{equation}
L = \mbox{max}\{L_i\} \quad R = \mbox{min}\{R_i\}.
\end{equation}

\textbf{[OUT-OF-CLASS]}
\begin{enumerate}[i.]
  \setcounter{enumii}{4}
\item Find $E(R-L)$. How does it depend on $n$?
\end{enumerate}

\textbf{[OPTIONAL]}: What is the name of this method? (Jun doesn't know.)




%%%%%%%%%%%%%%%%%%%%%%%%%%%%%%%%%%%%%%
\end{enumerate}
\end{document}

%%%%%%%%%%%%%%%%%%%%%%%%%%%%%%%%%%%%%%
